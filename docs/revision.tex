\documentclass[onecolumn,12pt,letterpaper]{article}

\usepackage[utf8]{inputenc}
\usepackage[spanish]{babel}
\usepackage[right=1.5cm,top=1cm,bottom=2cm,left=1.5cm]{geometry}
% \usepackage{mcite}

\title{Revisión sobre trabajos de crimen}
\author{pachichi}
\date{2019/08/23}

\begin{document}
\maketitle

\section{Reviews}

Uno de los artículos de revisión sobre modelos matemáticos y crimen es \cite{gordon_random_2010}. D'orsogna et. al.  \cite{dorsogna_statistical_2015} hacen una revisión de modelos sobre crimen usando metodos de la física estadística.

\section{Primarios}



Otro idea forma en que se puede analizar es de una forma similar al análisis cerebral por medio de ECG (Tesis de Octavio Lecona UACM). Es decir, de las colonias, cuadrantes de seguridad o delegaciones obtener las series de tiempo para algún tipo de delito, encontrar las correlaciones y formar una red. 


Sobre crimen en la CDMX esta el trabajo de Carlos Piña\cite{pina-garcia_exploring_2019} que por medio de tuits trata de predecir tendencias del crimen.
\section{Secundarios}






\bibliographystyle{apalike}
\bibliography{biblio.bib}


\end{document}
