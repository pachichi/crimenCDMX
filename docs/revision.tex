\documentclass[onecolumn,12pt,letterpaper]{article}

\usepackage[utf8]{inputenc}
\usepackage[spanish]{babel}
\usepackage[right=1.5cm,top=1cm,bottom=2cm,left=1.5cm]{geometry}
\usepackage{url}
% \usepackage[backend=bibtex,style=numeric,natbib=true]{biblatex}
% \bibliography{biblio}
\usepackage{mcite}

\title{Revisión bibliográfica sobre modelos matemáticos, crimen, mapas y datos}
\author{Sergio Sánchez}
\date{2019/08/23}

\begin{document}
\maketitle

Uno de los problemas que aqueja a nuestra sociedad es la cometición de delitos dentro de ella. Este es un problema cuyos factores que influencian el incremento o decremento son multiples. Lo ideal sería encontrar la causas que origian esos delitos, o por lo menos ser capaces de hacer una predicción. En particular me interesa encontrar alguna respuesta nueva o aplicar algún método existente, desde el enfoque de Sistemas Complejos, al caso particular de la ciudad de méxico, apoyandomé en las bases de datos que ha liberado, a traves de la ADIP, el gobierno de la ciudad sobre carpetas de inverstigación de la fiscalia de la cdmx (antes procuraduría). 

% Quizás desde la modelación de la complejidad no estemos buscando predecir las tendencias criminales o la aparición de hotspot, quizá lo importante es capturar alguna propiedad del sistema que este oculta. 
% La mayoría de los artículos tratan de predecir el comportamiento criminal. De estos datos podemos saber o por lomenos vislumbrar ¿cuáles son los mecanismos que se deben atacar para ? 


\section{Reviews}

Como punto de partida se puede consultar algunos reviews \mcite{gordon_random_2010,dorsogna_statistical_2015}. En el primero de ellos hacen una revisión de crimen y modelación matemática enfocados principalmente en modelos estadísticos. El propio artículo hace referencia a que la mayoría de los artículos desde 1970 se han basado en modelos lineales donde la tasa de criminalidad depende de una función lineal de variables que lo explican. Menciona dos tipos de modelos, económicos y sociales. D'orsogna et. al.\cite{dorsogna_statistical_2015} hacen un revisión sobre crimen y métodos de la física estadística, algunos de ellos son ecuaciones parciales diferenciales para la formación de patrones, self-exciting point process, teoría de redes, teoría de juegos y ABMs.

% en \cite{gordon_random_2010} hacen una observación sobre los datos obtenidos georeferenciados y que son proporcionados por los reportes de la policia. porque no son datos obtenidos para investigación pueden tener algunas imprecisiones. donde pueden tener errores no sistemáticos.


\section{Primarios}

un modelo sobre una malla de robo a casas que pueden llevar a un sistema de reacción-difusión. Detrás de este modelo esta la teoría de los vidrios rotos. \cite{short_statistical_2008}. 


Utiliza las series de tiempo de delitos para hacer un análisis fractal, también obtiene el exponente de Hurts\cite{melgarejo_multifractal_2017}.  


Otro idea forma en que se puede analizar es de una forma similar al análisis cerebral por medio de EEG (Octavio Lecona, Ruben Fossion). Es decir, de las colonias, cuadrantes de seguridad o delegaciones obtener las series de tiempo para algún tipo de delito, encontrar las correlaciones y formar una red. 


% \section{Secundarios}

Sobre este proceso self-exciting point process también se ha utilizado para predecir la popularidad de tuits\cite{zhao2015seismic}. En \cite{reinhart2018review} se hace una revisión de este método y las aplicaciones en las que se ha usado. 

Este también esta interesante porque utiliza estadística circular para analizar los delitos por hora.\cite{brunsdon2006using}.


En este hablan también de los calculos de la estadística espacial (spatial description, hot spot analysis, interpolation, space-time analysis, and journey-to.crime modeling), en particular de un programa llamado \textit{CrimeStat}\cite{levine2006crime}

Estos los que utilizan enfoque de redes sobre las calles\cite{rosser_predictive_2017}. \cite{porta_network_2006} \cite{porta_network_2006-1} \cite{davies_modelling_2013}

Estos son los artículos donde utilizan ABM; aquí también usan algoritmos genéticos\cite{furtado_bio-inspired_2009}, \cite{malleson_agent-based_2009}, \cite{malleson_crime_2010} , desplazamiento de hotspot por medio de ABM\cite{wang_analyzing_2014}, \cite{devia_generating_2013}. \cite{hegemann_geographical_2011}. \cite{groff_simulation_2007} también utiliza teoría de juegos  \cite{bruni_what_2013}.


Estos los que utilizan Redes Neuronales \cite{francisco_alisis_2015}, \cite{olligschlaeger_artificial_1998}
En el de Guayaquil se comenta desde como formar los hostpots con dos tipos de algoritmos, DBSCAN y k-Means. Utiliza un red neuronal artificial para predecir hotspots.\cite{garcia-plua_deteccion_2017} \cite{zhuang_crime_2017}. Grid level prediction. loistic regresion and neural networks. \cite{rummens_use_2017}

Modelo continuo de campo medio en un dimensión para robos a casa-habitación donde los ladrones se mueven siguiendo vuelos de Lécy truncados.\cite{pan_crime_2018}. Ecuaciones de reacción difusión \cite{short_dissipation_2010}, incorporar al modelo de RD la información de los datos con una probabilidad de densidad \cite{woodworth_j._t._non-local_2014}. Mas sobre el modelo de RD \cite{short_nonlinear_2010}


Libros sobre crimen y mapas. \cite{rossmo_geographic_2014} \cite{santos_crime_2016} \cite{liu_artificial_2008} \cite{weisburd_crime_1998} \cite{chainey_crime_2008}. 


Criminilidad y modelo de ising usan modelo propuesto por dorsogna \cite{ayouche_second_2015}, 
Modelo estadístico sobre pandillas que interactuan en un territorio a traves del graffiti.(transiciones de fase)\cite{barbaro_territorial_2013}

Hace enfásis en el mapeo de los delitos, también examina diferentes maneras en localizar hotspots, me parecen medidas bastante sencillas \cite{ratcliffe_crime_2010}. 


Cuando la cantidad de datos es grande, se pueden utilizar técnicas de \textit{Big Data}. En el trabajo de Hassani\cite{hassani_review_2016} examina varias técnicas de \textit{Big Data}, donde  la localización y análisis de hotspots la clasifica dentro de la técnica de \textit{Cluster Analysis}. En el último Reporte de Seguridad de la Ciudad de México (Agosto 2019)\footnote{\url{https://datosseguridad.cdmx.gob.mx/tablero/diario/clusters}}, se utilizó el algoritmo \textit{Density-based spatial clustering of applications with noise (DBSCAN)}, para formar \textit{clusters} sobre el mapa de la CDMX con los datos del crimen.\cite{noauthor_datos_nodate}.  \cite{noauthor_big_nodate} \cite{chen_crime_2004},


Este es de machine-learning \cite{alves_crime_2018}
Linear regression, logistic regression and gradient boosting \cite{ingilevich_crime_2018}. Series de tiempo y machine learning \cite{wang_learning_2013}


Enfoque estadistico\cite{noauthor_bayesian_nodate}, Bayesian methods and stochastic differential equation \cite{mohler_geographic_2012}. Probabilistico, point-pattern-based transition density model. hot spot prediction\cite{liu_criminal_2003} 

Enfoque de sistemas dinámicos como una epidemia \cite{gonzalez-parra_mathematical_2018} \cite{srivastav_modeling_2019} \cite{mcmillon_modeling_2014}


Medio viejon y habla sobre el hotspoting en general y los métodos tradicionales para llevarlo a acabo.\cite{bowers_prospective_2004}. Muy general y muy vago \cite{butorac_geography_2017}


El número de homicidios y la posición espacial así como el desarrollo en el tiempo como un fenoméno de percolación \cite{alves_spatial_2015} 

criminal inspector and ordinaru people. transiciones de fase \cite{perc_understanding_2013} 

Modelo económico sobre el crimen montado sobre ecuaicones en diferencias. \cite{freeman_spatial_1996}

\subsection{Crimen CDMX}
\label{sec:crimen-cdmx}

Sobre crimen en la CDMX esta el trabajo de Carlos Piña\cite{pina-garcia_exploring_2019} que por medio de tuits trata de predecir tendencias del crimen. Analisis de Crimen en el df, libro \cite{mendoza_tamano_2012}; artículos \cite{mata_mobile_2016}, \cite{vilalta_what_2016}, \cite{sanchez_salinas_robo_2016}, \cite{fuentes_flores_distribucion_2017}, \cite{cisneros_geografimiedo_2008}, Ciudad Juarez accidentes de tránsito\cite{hernandez_hernandez_alisis_2012}. Otro trabajo que quizás sea importante\cite{espinal-enriquez_analysis_2015}, arman un red con las correlaciones. A parte de los trabajos desde la parte académica 'pura', recientemente se ha realizado otro tipo de analisis por parte de la sociedad civil, y del mundo de la ciencia de  datos, principalmente usan un enfoque de ciencia de datos,   \cite{noauthor_delincuencia_nodate}
\cite{noauthor_crimen_nodate} \cite{noauthor_robo_nodate} \cite{noauthor_homicidios_nodate} \cite{noauthor_visualizar_nodate} \cite{que_nodate} \cite{infodatacivica.org_explorador_nodate} \cite{robo_nodate}




\bibliographystyle{unsrt}
\bibliography{biblio.bib}
% \printbibliography


\end{document}
